\documentclass[10pt, a4paper]{article}
\usepackage{lrec2022} % this is the new LREC2022 Style
\usepackage{multibib}
\newcites{languageresource}{Language Resources}
\usepackage{graphicx}
\usepackage{tabularx}
\usepackage{soul}
% for eps graphics
%%% References and Labels
%%% Reference labels without a punctuation 
% courtesy of Marc Schulder , uni Hamburg ****************
\usepackage{titlesec}
%\titleformat{\section}{\normalfont\large\bf\center}{\thesection.}{1em}{}
\titleformat{\section}{\normalfont\large\bfseries\center}{\thesection.}{1em}{}
\titleformat{\subsection}{\normalfont\SmallTitleFont\bfseries\raggedright}{\thesubsection.}{1em}{}
\titleformat{\subsubsection}{\normalfont\normalsize\bfseries\raggedright}{\thesubsubsection.}{1em}{}
\renewcommand\thesection{\arabic{section}}
\renewcommand\thesubsection{\thesection.\arabic{subsection}}
\renewcommand\thesubsubsection{\thesubsection.\arabic{subsubsection}}
%  ed 

\usepackage{epstopdf}
\usepackage[utf8]{inputenc}

\usepackage{hyperref}
\usepackage{xstring}

\usepackage{color}

\newcommand{\secref}[1]{\StrSubstitute{\getrefnumber{#1}}{.}{ }}

\title{DaLUKE: Strengthening Danish NLP Using Weak Knowledge-Enhancement}

\name{Søren Winkel Holm, Asger Laurits Schultz, ...} 

\address{Technical University of Denmark, 2800 Kgs. Lyngby\\
         \{s183911, s183912, ...\}@dtu.dk\\}


\abstract{
    t
\newline \Keywords{NLP, Transformers, Entity-based Learning}
}

\begin{document}

\maketitleabstract

\section{Introduction}
The use of emerging languages technologies in industry has grown with positive impact in bussines and society, but this impact is largest in high-resource languages such as English motivating a focus on developing diverse methods for languages with much less available textual data \cite{hedderich2021survey}.
Languages such as Danish, with around 6 million speakers, fall in a size class where NLP ressources and an NLP tradition exist but with data of so limited magnitude that models such as large-scale transformers underperform.
In such domains, recent advances in combining explicit modeling of language knowledge with flexible SOTA models can be the key to get closer to the glory of the English-speaking models.
DaLUKE is such an attempt, using the general model idea of LUKE \cite{yamada2020luke} and adding custom solutions for the case of Danish to measure the potential and applicability of this combination.


\subsection{Related Models}
\paragraph{Knowledge-based Pretraining Augmentations}
\begin{itemize}
    \item KEPLER
    \item WKLM
    \item KALM
    \item K-Adapter
    \item KGpLM
    \item KB-VLP
    \item CoLAKE
\end{itemize}
\paragraph{Danish NLP}
\begin{itemize}
    \item DaBERT
    \item DaCy
\end{itemize}


\section{Methods}

\section{Results}

\section{Discussion}

\section{Bibliographical References}\label{reference}

\bibliographystyle{lrec2022-bib}
\bibliography{references}

\section{Language Resource References}
\label{lr:ref}
\bibliographystylelanguageresource{lrec2022-bib}
\bibliographylanguageresource{languageresource}

\end{document}
