% !TeX spellcheck = en_GB
% !BIB TS-program = biber
\documentclass[12pt, fleqn]{article}

\usepackage[utf8]{inputenc}
\usepackage[english]{babel}

\PassOptionsToPackage{hyphens}{url}\usepackage{hyperref}
\usepackage[top=2.5cm, bottom=2.5cm, left=3cm, right=3cm, includeheadfoot]{geometry}
\usepackage{fancyhdr}
\usepackage{graphicx}
\usepackage{float}
\usepackage{changepage}
\usepackage[nottoc, numbib]{tocbibind}
\usepackage{lastpage}
\usepackage{setspace}
\usepackage[bottom]{footmisc}
\usepackage{float}

\usepackage{amsmath}
\usepackage{amssymb}
\usepackage{nicefrac}
\usepackage{icomma}

\usepackage{color} %red, green, blue, yellow, cyan, magenta, black, white
\usepackage[dvipsnames]{xcolor}
\usepackage{titlesec}
\usepackage{listings}
\usepackage{multirow}

\usepackage{textcomp} % To avoid annoying \perthousand, \micro warnings
\usepackage{subfiles}
\usepackage{csquotes} % To avoid biber warnings
\usepackage[backend=biber, style=alphabetic, citestyle=alphabetic, maxcitenames=4, maxbibnames=4, mincitenames=2]{biblatex}
\usepackage{tabularx}

\usepackage{pgf,tikz}
\usetikzlibrary{arrows}
\usetikzlibrary{decorations.markings}

\usepackage{enumitem}
\setlist{noitemsep}

\allowdisplaybreaks

\fancypagestyle{plain}
{
    \fancyhf{}
    \rfoot{Page \thepage{}~of  \pageref{LastPage}}
    \renewcommand{\headrulewidth}{0pt}
}
\pagestyle{fancy}
\fancyhf{}

\lstset{frame=None,
    language={python},
    inputencoding=ansinew,
    literate=
    {æ}{{\ae}}1
    {å}{{\aa}}1
    {ø}{{\o}}1
    {Æ}{{\AE}}1
    {Å}{{\AA}}1
    {Ø}{{\O}}1,
    aboveskip=3mm,
    belowskip=3mm,
    showstringspaces=false,
    columns=flexible,
    basicstyle={\footnotesize\ttfamily},
    numbers=left,
    numberstyle=\tiny\color{gray}\ttfamily,
    keywordstyle=\color{blue}\ttfamily,
    ndkeywordstyle=\color{blue}\ttfamily,
    commentstyle=\color{gray}\ttfamily,
    stringstyle=\color{OliveGreen}\ttfamily,
    breaklines=true,
    breakatwhitespace=true,
    tabsize=4,
    escapeinside={<@}{@>},
    lineskip={-1.5pt},
    xleftmargin=1cm,
    xrightmargin=1cm
}

% Helper commands
\newcommand{\code}[1]{{\texttt{\small#1}}}
\newcommand{\numberthis}{\addtocounter{equation}{1}\tag{\theequation}}
\newcommand{\acomm}[1]{\hspace{2.5cm}\text{#1}}
\newcommand{\low}[1]{\ensuremath{_\textup{#1}}}

\newcommand{\andim}{\textup{ and }}
\newcommand{\raq}{\Rightarrow\quad}
\newcommand{\lraq}{\Leftrightarrow\quad}
\newcommand{\qandq}{\quad\wedge\quad}
\newcommand{\qorq}{\quad\vee\quad}
\newcommand{\diff}[2]{\ensuremath{\frac{\md #1}{\md #2}}}
\newcommand{\md}{\ensuremath{\text{d}}}

\newcommand{\ctp}[1]{\ensuremath{\cdot10^{#1}}}
\newcommand{\reci}{\ensuremath{^{-1}}}
\newcommand{\twopow}{\ensuremath{^{2}}}
\newcommand{\re}[1]{\ensuremath{^{#1}}}

\newcommand{\me}{\ensuremath{\operatorname{e}}}
\newcommand{\eul}[1]{\ensuremath{\me^{#1}}}
\newcommand{\len}[1]{\ensuremath{\left\lvert#1\right\rvert}}
\newcommand{\half}{\ensuremath{\frac{1}{2}}}
\newcommand{\third}{\ensuremath{\frac{1}{3}}}
\newcommand{\fourth}{\ensuremath{\frac{1}{4}}}
\newcommand{\transpose}[1]{\ensuremath{#1^{\textup T}}}

\newcommand{\NN}{\ensuremath{\mathbb N}}
\newcommand{\ZZ}{\ensuremath{\mathbb Z}}
\newcommand{\QQ}{\ensuremath{\mathbb Q}}
\newcommand{\RR}{\ensuremath{\mathbb R}}
\newcommand{\CC}{\ensuremath{\mathbb C}}
\newcommand{\LL}{\ensuremath{\mathbb L}}
\newcommand{\PP}{\ensuremath{\mathbb P}}

\newcommand{\unit}[1]{\ensuremath{\:\text{#1}}}
\newcommand{\pro}{\ensuremath{\unit{\%{}}}}

%Kommandoer til ændring af ligestillingsmargner
\newcommand{\jl}[1]{\multicolumn{1}{l}{#1}}
\newcommand{\jc}[1]{\multicolumn{1}{c}{#1}}
\newcommand{\jr}[1]{\multicolumn{1}{r}{#1}}
\newcommand{\jls}[1]{\multicolumn{1}{l|}{#1}}
\newcommand{\jcs}[1]{\multicolumn{1}{c|}{#1}}
\newcommand{\jrs}[1]{\multicolumn{1}{r|}{#1}}



\addbibresource{references.bib}

\chead{}
\rhead{Technical University of Denmark}
\rfoot{Page \thepage{}~of \pageref{LastPage}}

\graphicspath{{imgs/}{../imgs/}}

\title{DaLUKE: Strengthening Danish NLP Using Weak Knowledge-Enhancement}
\author{Søren Winkel Holm, Asger Laurits Schulz}
\date{\today}
\linespread{1.15}

\begin{document}
% Avoid warnings
\setlength{\headheight}{15pt}
\addtolength{\topmargin}{-2.5pt}
\maketitle

\begin{abstract}
    \begin{itemize}
        \item Fokusér på problemet med NLP i middelressourcesprog generelt
        \item Dansk er et case study, hvor vi undersøger LUKE-arkitekturen og vidensenhancement generelt som mulig løsning
        \item Inkludér reklame for kode
    \end{itemize}
\end{abstract}
\section{Introduction}%
\label{sec:Introduction}
\begin{itemize}
    \item Kort gennemgang af status i dansk NLP
    \item Kort introduktion af LUKE + nævn andre vidensenhancement-metoder
    \item Det mest generelle, motiverende spørgsmål: Hvad skal der til for at få det elegant LUKE-ideal til at virke med meget mere begrænset data?
    \item Mindre ressourcestærkt sprog => mindre ressourcestærke anvendere, så fokus på lille model
\end{itemize}

\section{Methods}%
\label{sec:Methods}

\begin{itemize}
    \item Data: Augmentering og forskellige kilder
    \item Hovedmodel: Hvad var forskellig fra DaLUKE
    \item Fremgangsmåde for træning
    \item Lille model: Fremgangsmåde for destillering/pruning
    \item Præsentér vores nye arkitektur til lavere entitetsdimension
    \item Præsentér NER-opgaven
\end{itemize}


\begin{table}[H]
    \centering
    \footnotesize
        \begin{tabular}{l|ll}
            Name                & Data & Base Model\\
            \hline
            Control             & Da. Wiki. w. entity links, Da. Gigaword & RoBERTa Base \\
            Auto-annotated      & Da. Wiki. w. entity links, Da. Gigaword with auto. entities & RoBERTa Base\\
            Big model           & Da. Wiki. w. entity links, Da. Gigaword & RoBERTa Large\\
            Low entity dim.     & Da. Wiki. w. entity links, Da. Gigaword & 252-dimensional ent., RoBERTa Base 
        \end{tabular}
    \caption{
        Learning rate: \(1.2\ctp{-4}\), batch size: 8160 examples
    }
    \label{tab:experiments}
\end{table}\noindent

\section{Results}%
\label{sec:Results}
\begin{itemize}
    \item Prætræningsperformance med forskellige modeller
    \item NER-performance sammenlignet med dansk niveau
\end{itemize}

\section{Discussion}%
\label{sec:Discussion}
\begin{itemize}
    \item Prætrænings-eksperimenter: Data, arkitektur og transferlæring kontrolleret
    \item En smule undersøgelse af maskeret opgave/downstream fejl
    \item Konklusioner: AI-succes bliver bestemt af nogle andre underliggende parametre i højressourcedomænet end i den dataknappe situation
\end{itemize}

% Smaller bib text
\renewcommand*{\bibfont}{\normalfont\footnotesize}
\printbibliography[heading=bibintoc]

\appendix

%\begin{tabular}[t]{l>{}l>{}l>{}l>{}l>{}l>{}l>{}l}
%\multicolumn{1}{c}{ } & \multicolumn{7}{c}{Stochastic Augmentations} \\
%\multicolumn{1}{c}{ } & \multicolumn{4}{c}{Names} & \multicolumn{3}{c}{Keystroke Errors} \\
%Model & Danish & Muslim & Female & Male & 2\% & 5\% & 15\%\\
%DaLUKE & \textbf{84.3 (0.6)} & \textbf{82.1 (0.9)*} & \textbf{85.2 (0.6)*} & \textbf{84.3 (0.8)} & \textbf{71.1 (1.6)*} & \textbf{57.3 (1.7)*} & \textbf{33.0 (1.7)*}\\
%\end{tabular}


\end{document}

